\documentclass[UTF8]{pkuthss}
\usepackage[backend = biber, style = caspervector, utf8,
sorting = none, giveninits = true, sortgiveninits = true
]{biblatex}
\usepackage{graphicx}
\usepackage{amsmath}
\usepackage{url}
\usepackage{hyperref}
\DefineBibliographyStrings{english}{volume= {Vol.}}
\setlength{\hfuzz}{3pt}
\ctexset{linestretch=2\ccwd}
\setlength{\bibitemsep}{3bp}
\hypersetup{colorlinks=true, allcolors=cyan}
\usepackage{simpler-wick}
\usepackage[figuresright]{rotating}
\usepackage{tikz,tikz-feynman}
\usepackage{ulem}
\usepackage{simplewick}
\usepackage{makecell}
\usepackage{multirow}
% \usepackage{braket}
\usepackage{enumitem}
% 量子力学 bra ket 符号
\newcommand{\bra}[1]{\langle{#1}|}
\newcommand{\ket}[1]{|{#1}\rangle}
\newcommand{\braket}[2]{\langle{#1}|{#2}\rangle}
% 尖括号平均值
\newcommand{\anglemean}[1]{\langle{#1}\rangle}
\DeclareMathOperator{\Tr}{Tr}

% 用来设置附录中代码的样式
% \usepackage{listings}
% \lstset{
% basicstyle          =   \sffamily,          % 基本代码风格
% keywordstyle        =   \bfseries,          % 关键字风格
% commentstyle        =   \rmfamily\itshape,  % 注释的风格,斜体
% stringstyle         =   \ttfamily,  % 字符串风格
% flexiblecolumns,                % 别问为什么,加上这个
% numbers             =   left,   % 行号的位置在左边
% showspaces          =   false,  % 是否显示空格,显示了有点乱,所以不现实了
% numberstyle         =   \zihao{-5}\ttfamily,    % 行号的样式,小五号,tt等宽字体
% showstringspaces    =   false,
% captionpos          =   t,      % 这段代码的名字所呈现的位置,t指的是top上面
% frame               =   lrtb,   % 显示边框
% }
% \lstdefinestyle{Python}{
% language        =   Python, % 语言选Python
% basicstyle      =   \zihao{-5}\ttfamily,
% numberstyle     =   \zihao{-5}\ttfamily,
% keywordstyle    =   \color{blue},
% keywordstyle    =   [2] \color{teal},
% stringstyle     =   \color{magenta},
% commentstyle    =   \color{red}\ttfamily,
% breaklines      =   true,   % 自动换行,建议不要写太长的行
% columns         =   fixed,  % 如果不加这一句,字间距就不固定,很丑,必须加
% basewidth       =   0.5em,
% }
% \lstdefinestyle{C++}{
% language        =   C++, % 语言选C++
% basicstyle      =   \zihao{-5}\ttfamily,
% numberstyle     =   \zihao{-5}\ttfamily,
% keywordstyle    =   \color{cyan},
% keywordstyle    =   [2] \color{teal},
% stringstyle     =   \color{magenta},
% commentstyle    =   \color{green}\ttfamily,
% breaklines      =   true,   % 自动换行,建议不要写太长的行
% columns         =   fixed,  % 如果不加这一句,字间距就不固定,很丑,必须加
% basewidth       =   0.5em,
% }
% \lstinputlisting[style=C++,caption={\bf apwp.hpp},label={apwp.hpp}]{./code/apwp.hpp} % 一个例子

\pkuthssinfo{
cthesisname = {博士研究生学位论文}, ethesisname = {Doctor Thesis},
ctitle = {基于全组态量子蒙特卡洛的核物理第一性原理精确计算},
etitle = {Full-configuration interaction quantum Monte Carlo},
cauthor = {胡荣哲},
eauthor = {Rongzhe Hu},
studentid = {2101110124},
date = {二零二六年五月},
school = {物理学院},
cmajor = {粒子物理与原子核物理}, emajor = {Particle and Nuclear Physics},
direction = {理论核结构},
cmentor = {许甫荣 教授}, ementor = {Prof.\ Furong Xu},
ckeywords = {第一性原理;量子蒙特卡洛;手征核力}, ekeywords = {\textit{ab initio}, quantum Monte Carlo, chiral nuclear force}
}
\normalem
\addbibresource{reference.bib}


\begin{document}

\frontmatter
\pagestyle{empty}
\maketitle
\cleardoublepage
\input{src-app/copyright}
\cleardoublepage
\pagestyle{plain}
\setcounter{page}{0}
\pagenumbering{Roman}
\begin{cabstract}
    原子核结构的第一性原理计算({\it ab initio})是指从现实核力出发,运用严格的量子多体方法求解原子核结构问题。本论文对多体微扰理论及壳模型计算进行了如下发展:
\begin{enumerate}
    \item \textbf{全组态相互作用量子蒙特卡洛}\; 壳模型计算是在一个较大的组态空间对哈密顿量直接对角化,由于组态空间维度随粒子增加而迅速增大,其难以计算较多核子数的体系,比如无核芯壳模型仅适用于轻核的计算。全组态相互作用量子蒙特卡洛(full-configuration interaction Quantum Monte Carlo, FCIQMC)方法是在电子结构计算中获得成功的一种组态空间计算方法,其通过对组态空间波函数进行蒙特卡洛采样,能够一定程度上解决组态空间维度过大的问题。我们成功将这一方法应用到原子核结构计算中,并通过$^{56}\mathrm{Ni}$的计算,表明这一方法能够比耦合团簇方法考虑更多的多体关联。
\end{enumerate}
\end{cabstract}

\begin{eabstract}
First-principles ({\it ab initio}) nuclear structure calculations refer to solving nuclear structure problems using rigorous quantum many-body methods starting from realistic nuclear forces. This thesis presents the following advancements in MBPT and shell model calculations:

\begin{enumerate}
    \item \textbf{Full-Configuration Interaction Quantum Monte Carlo}\; The shell model involves direct diagonalization of Hamiltonians in large configuration spaces. Due to rapid dimension growth with increasing nucleon numbers, it becomes challenging for systems with more nucleons (e.g., no-core shell models are limited to light nuclei). The full-configuration interaction Quantum Monte Carlo (FCIQMC) method, successful in electronic structure calculations, addresses the issue of configuration space dimensionality through Monte Carlo sampling of the configuration-space wavefunction. We successfully implemented this method in nuclear structure calculations. Through $^{56}\mathrm{Ni}$ calculations, we demonstrated that FCIQMC can account for more extensive many-body correlations compared to coupled-cluster methods.

\end{enumerate}
\end{eabstract}
\tableofcontents
\mainmatter


\chapter{绪论}\label{chap:intro}

原子核是一个自束缚的量子多体系统。

\chapter{核力的基本理论}\label{chap:nuclear-force}

核子系统独特的复杂性来源于核力的复杂性。从现代的物理学观点我们知道,核力的本质是量子色动力学(QCD),体现为强相互作用在原子核体系中的剩余相互作用,而QCD的低能非微扰特性更是为核力蒙上了一层神秘的面纱。从卢瑟福实验开始,对核力理论的探索贯穿了整个近代物理学的发展历史,直到现在仍是一个极具挑战性的前沿方向。

对各种实验结果的初步分析,我们可以得知核力应该具有以下基本性质:

\begin{enumerate}
\item 核力是短程强相互作用,有效力程小于3 fm。
\item 除了中心力外,核力具有明显的自旋相关部分。
\item 核力具有短程排斥芯,当两核子距离小于0.4 fm时有强烈的排斥势。
\item 核力具有近似的电荷无关性。
\end{enumerate}

我们将会在后面看到,手征核力能够自然满足这些基本性质,无需任何额外人为假设。

\section{核力的一般形式}
核力是一个标量算符,其表达式由动量、自旋和同位旋部分组成,其具有的平移对称、时间反演、空间反射和空间旋转对称性极大地约束了核力的可能算符结构。

\subsection{两体核力的算符结构}
对于两体核力,可以证明其在自旋空间最多具有6个线性无关的算符结构[\cite{PhysRev.96.1654}],最一般形式可以写为:

\begin{equation}
    \begin{aligned}
        \hat{w}_1&=1\\
        \hat{w}_2&=\boldsymbol{\sigma}_1\cdot\boldsymbol{\sigma}_2\\
        \hat{w}_3&=(\boldsymbol{\sigma}_1\cdot\boldsymbol{q})(\boldsymbol{\sigma}_2\cdot\boldsymbol{q})\\
        \hat{w}_4&=(\boldsymbol{\sigma}_1\cdot\boldsymbol{k})(\boldsymbol{\sigma}_2\cdot\boldsymbol{k})\\
        \hat{w}_5&=-\mathrm{i}\boldsymbol{S}\cdot(\boldsymbol{q}\times\boldsymbol{k})\\
        \hat{w}_6&=[\boldsymbol{\sigma}_1\cdot(\boldsymbol{q}\times\boldsymbol{k})][\boldsymbol{\sigma}_2\cdot(\boldsymbol{q}\times\boldsymbol{k})],
    \end{aligned}
\end{equation}
其中我们约定:

\begin{enumerate}
\item $\boldsymbol{k}_1,\boldsymbol{k}_2(\boldsymbol{k}'_1,\boldsymbol{k}'_2)$:实验室系下两个粒子的单粒子初(末)动量
\item $\boldsymbol{p}=\boldsymbol{k}_1-\boldsymbol{k}_2(\boldsymbol{p}'=\boldsymbol{k}'_1-\boldsymbol{k}'_2)$:质心系下的两粒子初(末)相对动量
\item $\boldsymbol{K}=\boldsymbol{k}_1+\boldsymbol{k}_2(\boldsymbol{K}'=\boldsymbol{k}'_1+\boldsymbol{k}'_2)$:实验室系下两粒子初(末)总动量
\item $\boldsymbol{q}=\boldsymbol{p}'-\boldsymbol{p}$:初末态动量转移
\item $\boldsymbol{k}=(\boldsymbol{p}'+\boldsymbol{p})/2$:平均(转移)动量
\item $\boldsymbol{\sigma}_{1,2},\boldsymbol{\tau}_{1,2}$:自旋、同位旋算符
\item $\boldsymbol{S}=(\boldsymbol{\sigma}_1+\boldsymbol{\sigma}_2)/2$:总自旋算符
\end{enumerate}

那么两体核力可以展开为:

\begin{equation}
\hat{V}_{\mathrm{NN}}(\boldsymbol{p}',\boldsymbol{p}) = \sum_{i=1}^{6} f_i(\boldsymbol{p}',\boldsymbol{p}) \hat{w}_i,
\end{equation}
前面的$f_i(\boldsymbol{p}',\boldsymbol{p})$是与自旋无关的标量函数。

如果考虑的是散射矩阵元,能量守恒要求$\boldsymbol{k}\cdot\boldsymbol{q}=0$和$(\boldsymbol{k}\times\boldsymbol{q})^2=\boldsymbol{k}^2\cdot\boldsymbol{q}^2$,借助如下恒等式:
\begin{equation}
\begin{aligned}
&(\boldsymbol{\sigma}_1\cdot\boldsymbol{k})(\boldsymbol{\sigma}_2\cdot\boldsymbol{k})\cdot\boldsymbol{q}^2 + (\boldsymbol{\sigma}_1\cdot\boldsymbol{q})(\boldsymbol{\sigma}_2\cdot\boldsymbol{q})\cdot\boldsymbol{k}^2 + (\boldsymbol{\sigma}_1\cdot\boldsymbol{k}\times\boldsymbol{q})(\boldsymbol{\sigma}_2\cdot\boldsymbol{k}\times\boldsymbol{q})\\
&= (\boldsymbol{\sigma}_1\cdot\boldsymbol{\sigma}_2)(\boldsymbol{k}\times\boldsymbol{q})^2 + (\boldsymbol{k}\cdot\boldsymbol{q})\left[ (\boldsymbol{\sigma}_1\cdot\boldsymbol{k})(\boldsymbol{\sigma}_2\cdot\boldsymbol{q}) + (\boldsymbol{\sigma}_1\cdot\boldsymbol{q})(\boldsymbol{\sigma}_2\cdot\boldsymbol{k}) \right],
\end{aligned}
\end{equation}
可以将独立的自旋算符结构减少为5个。

最终,再考虑到同位旋算符,最终两体核力的算符结构为[\cite{MACHLEIDT20111}]:

\begin{equation}
\begin{aligned}
\hat{V}_{\mathrm{NN}}(\boldsymbol{p}',\boldsymbol{p})= & \quad V_C+\boldsymbol{\tau}_1 \cdot \boldsymbol{\tau}_2 W_C\\
&+\left[V_S+\boldsymbol{\tau}_1 \cdot \boldsymbol{\tau}_2 W_S\right] \boldsymbol{\sigma}_1\cdot\boldsymbol{\sigma}_2 \\
& +\left[V_{L S}+\boldsymbol{\tau}_1 \cdot \boldsymbol{\tau}_2 W_{L S}\right] \left[-\mathrm{i}\boldsymbol{S}\cdot(\boldsymbol{q}\times\boldsymbol{k}) \right]\\
& +\left[V_T+\boldsymbol{\tau}_1 \cdot \boldsymbol{\tau}_2 W_T\right] (\boldsymbol{\sigma}_1\cdot\boldsymbol{q})(\boldsymbol{\sigma}_2\cdot\boldsymbol{q}) \\
& +\left[V_{\sigma L}+\boldsymbol{\tau}_1 \cdot \boldsymbol{\tau}_2 W_{\sigma L}\right] \boldsymbol{\sigma}_1\cdot(\boldsymbol{q}\times\boldsymbol{k})\boldsymbol\,{\sigma}_2\cdot(\boldsymbol{q}\times\boldsymbol{k}).
\end{aligned}
\end{equation}
记$\boldsymbol{\tau}_1 \cdot \boldsymbol{\tau}_2$的矩阵元为$\mathcal{T}$,对比我们之前的定义,有如下对应关系:
\begin{equation}
\begin{aligned}
V_C,W_C &\to f_1,\mathcal{T}f_1\\
V_S,W_S &\to f_2,\mathcal{T}f_2\\
V_{LS},W_{LS} &\to f_5,\mathcal{T}f_5\\
V_{T},W_{T} &\to f_3,\mathcal{T}f_3\\
V_{\sigma L},W_{\sigma L} &\to f_6,\mathcal{T}f_6.
\end{aligned}
\end{equation}


\subsection{三体核力的算符结构}
与两体核力类似,三体核力最初写在相对的Jacobi坐标下。我们定义:
\begin{enumerate}
\item $\boldsymbol{k}_1,\boldsymbol{k}_2,\boldsymbol{k}_3(\boldsymbol{k}'_1,\boldsymbol{k}'_2,\boldsymbol{k}'_3)$:实验室系下3个粒子的单粒子初(末)动量
\item $\boldsymbol{q}_i=\boldsymbol{k}'_i-\boldsymbol{k}_i(i=1,2,3)$:第$i$个粒子的初末动量变化
\item $\boldsymbol{p},\boldsymbol{q},\boldsymbol{p}',\boldsymbol{q}'$:初末Jacobi动量
\item $\boldsymbol{K}=\boldsymbol{k}_1+\boldsymbol{k}_2+\boldsymbol{k}_3(\boldsymbol{K}')$:三体初(末)总动量,是守恒量
\end{enumerate}
动量的变换关系为:
\begin{equation}
\begin{aligned}
\boldsymbol{q}_1&=\boldsymbol{q}'-\boldsymbol{q}\\
\boldsymbol{q}_2&=\boldsymbol{p}'-\dfrac{1}{2}\boldsymbol{q}'-\Big(\boldsymbol{p}-\dfrac{1}{2}\boldsymbol{q}\Big)\\
\boldsymbol{q}_3&=-\boldsymbol{p}'-\dfrac{1}{2}\boldsymbol{q}'-\Big(-\boldsymbol{p}-\dfrac{1}{2}\boldsymbol{q}\Big)
\end{aligned}
\end{equation}

对于三体核力,仅仅利用对称性进行约束时,可能的算符结构数目会变得非常多,因此基本无法写出一个普遍的表达式。幸运的是,在手征有效场论的框架下,这些都可以逐阶得到。例如在N$^2$LO阶,出现了如下4种算符结构[\cite{HEBELER20211}]:
\begin{equation}
\begin{aligned}
V_{\text{3N}}=\sum_{(i,j,k)\in \mathcal{I}} \Big[ &
F_{\text{TPE1}}^{ij}(\boldsymbol{\sigma}_i\cdot \boldsymbol{q}_i) \;(\boldsymbol{\sigma}_j\cdot \boldsymbol{q}_j)\;(\boldsymbol{\tau}_i \cdot \boldsymbol{\tau}_j)\\
+&F_{\text{TPE2}}^{ij}(\boldsymbol{\sigma}_i\cdot \boldsymbol{q}_i) \;(\boldsymbol{\sigma}_j\cdot \boldsymbol{q}_j)\;\boldsymbol{\sigma}_k\cdot (\boldsymbol{q}_i\times \boldsymbol{q}_j)
\;\boldsymbol{\tau}_k\cdot (\boldsymbol{\tau}_i\times \boldsymbol{\tau}_j)\\
+&F_{\text{OPE}}^{j}(\boldsymbol{\sigma}_j\cdot \boldsymbol{q}_j)\;(\boldsymbol{\sigma}_i\cdot \boldsymbol{q}_j)\;(\boldsymbol{\tau}_i \cdot \boldsymbol{\tau}_j)\\
+&F_{\text{contact}}(\boldsymbol{\tau}_j \cdot \boldsymbol{\tau}_k)
\Big],
\end{aligned}
\end{equation}
其中指标对$i\neq j\neq k$循环求和:
\begin{equation}
\mathcal{I}=\{(123),(132),(213),(312),(231),(321)\}
\end{equation}


\section{表象与矩阵元}
根据是否进行分波分解,核力的表象分为平面波表象(plane-wave basis)和分波表象(partial-wave basis, LSJ basis, spherical-wave basis),当然也有诸如螺旋态表象(helicity basis)等表象。核力矩阵元在上述不同表象之间通过正交变换相互联系。动量表象与坐标表象之间则通过傅立叶变换互相联系。在有限核的多体计算中,核力矩阵元需要进一步转到谐振子表象下。

\subsection{平面波表象}
单粒子自由薛定谔方程的解就是单粒子平面波波函数,对于某一个$i$核子来说,其单体态记为:
\begin{equation}
|i\rangle \equiv
|\boldsymbol{k}_i\sigma_i\tau_i\rangle \equiv |\boldsymbol{k}_i\rangle \otimes |\sigma_i\rangle \otimes |\tau_i\rangle,
\end{equation}
$A$体直积态也类似定义为:
\begin{equation}
|ij...A\rangle \equiv |i\rangle \otimes |j\rangle \otimes ... \otimes |A\rangle.
\end{equation}
比如两体态和三体态:
\begin{equation}
\begin{aligned}
|12\rangle&=|\boldsymbol{k}_1\sigma_1\tau_1\boldsymbol{k}_2\sigma_2\tau_2\rangle,\\
|123\rangle&=|\boldsymbol{k}_1\sigma_1\tau_1\boldsymbol{k}_2\sigma_2\tau_2\boldsymbol{k}_3\sigma_3\tau_3\rangle.
\end{aligned}
\end{equation}

下面我们总结核力的两体、三体平面波矩阵元的流程。
\begin{enumerate}[label=\sffamily\textbf{[\arabic*]}, noitemsep]

\item 利用Mathematica等符号计算软件求出相对坐标系下的矩阵元表达式,
\begin{equation}
\begin{aligned}
\big\langle\boldsymbol{p}'\sigma'_1\tau'_1\sigma'_2\tau'_2 \big|&V_{\mathrm{NN}} \big|\boldsymbol{p}\sigma_1\tau_1\sigma_2\tau_2\big\rangle,\\
\big\langle\boldsymbol{p}'\boldsymbol{q}'\sigma'_1\tau'_1\sigma'_2\tau'_2\sigma'_3\tau'_3 \big|&V_{\mathrm{3N}} \big|\boldsymbol{p}\boldsymbol{q}\sigma_1\tau_1\sigma_2\tau_2\sigma_3\tau_3\big\rangle.
\end{aligned}
\end{equation}
注意上面两式中的$\boldsymbol{p}$定义不同,一个是两核子相对动量,一个是Jacobi动量。

\item 借助单体态的正交归一关系$\langle\boldsymbol{K}'|\boldsymbol{K}\rangle=\delta(\boldsymbol{K}'-\boldsymbol{K})/L^3$将矩阵元转到实验室系:
\begin{equation}
\begin{aligned}
&\langle\boldsymbol{k}'_1\sigma'_1\tau'_1\boldsymbol{k}'_2\sigma'_2\tau'_2|V_{\mathrm{NN}}|\boldsymbol{k}_1\sigma_1\tau_1\boldsymbol{k}_2\sigma_2\tau_2 \rangle \\ =\;&\langle\boldsymbol{p}'\sigma'_1\tau'_1\sigma'_2\tau'_2|V_{\mathrm{NN}}|\boldsymbol{p}\sigma_1\tau_1\sigma_2\tau_2\rangle
\times\dfrac{1}{L^3}\delta(\boldsymbol{K}'-\boldsymbol{K}) \delta_{\tau'_1+\tau'_2,\tau_1+\tau_2},
\end{aligned}
\end{equation}
\begin{equation}
\begin{aligned}
&\langle\boldsymbol{k}'_1\sigma'_1\tau'_1\boldsymbol{k}'_2\sigma'_2\tau'_2\boldsymbol{k}'_3\sigma'_3\tau'_3|V_{\mathrm{3N}}|\boldsymbol{k}_1\sigma_1\tau_1\boldsymbol{k}_2\sigma_2\tau_2\boldsymbol{k}_3\sigma_3\tau_3 \rangle \\ =\;&\langle\boldsymbol{p}'\boldsymbol{q}'\sigma'_1\tau'_1\sigma'_2\tau'_2\sigma'_3\tau'_3|V_{\mathrm{3N}}|\boldsymbol{p}\boldsymbol{q}\sigma_1\tau_1\sigma_2\tau_2\sigma_3\tau_3\rangle 
\times\dfrac{1}{L^3}\delta(\boldsymbol{K}'-\boldsymbol{K}) \delta_{\tau'_1+\tau'_2+\tau'_3,\tau_1+\tau_2+\tau_3}.
\end{aligned}
\end{equation}

\item 进行反对称化,其中反对称化算符:
\begin{equation}
\mathcal{A}_{12}=1-\mathcal{P}_{12},
\end{equation}
\begin{equation}
\mathcal{A}_{123}=1-\mathcal{P}_{12}-\mathcal{P}_{13}-\mathcal{P}_{23}+\mathcal{P}_{123}+\mathcal{P}_{132}.
\end{equation}
最后得到反对称化的矩阵元:
\begin{equation}
\bra{12}\bar{V}_{\mathrm{NN}}\ket{34}=\bra{12}V_{\mathrm{NN}}\ket{34}-\bra{12}V_{\mathrm{NN}}\ket{43},
\end{equation}
\begin{equation}
\begin{aligned}
\bra{123}\bar{V}_{\mathrm{3N}}\ket{456} =\; &\bra{123}V_{\mathrm{3N}}\ket{456} -\bra{123}V_{\mathrm{3N}}\ket{546} -\bra{123}V_{\mathrm{3N}}\ket{654} \\
-&  \bra{123}V_{\mathrm{3N}}\ket{465} +\bra{123}V_{\mathrm{3N}}\ket{645} +\bra{123}V_{\mathrm{3N}}\ket{564}.
\end{aligned}
\end{equation}
在之后的多体计算中,使用的矩阵元默认都是经过反对称化的,因此以后不做区分。

\end{enumerate}

对于两体核力,可以利用Mathematica把算符结构的自旋同位旋矩阵元的解析表达式直接求出来,之后对于所有的核力形式只需要将前面的标量函数成分辨认出来即可,为编写高性能计算程序提供了极大便利。我们将这种方法称为自动化平面波投影(automated plane-wave projection, aPWP),相关代码见[\cite{code-apwp}]。


\subsection{两体分波表象}
分波表象是核力理论中最常使用的表象,也是通往多体计算的必经之路,其基本思想就是量子力学散射问题中的分波法。对于有自旋的两粒子系统,我们可以选出力学完备集$\{ \hat{H}_0, \boldsymbol{J}^2, \boldsymbol{J}_z, \boldsymbol{L}^2, \boldsymbol{S}^2, \boldsymbol{S}_1^2, \boldsymbol{S}_2^2\}$, 其中$\hat{H}_0=\boldsymbol{p}^2/2\mu$是相对动能、$\boldsymbol{J}^2$是总角动量、$\boldsymbol{L}^2$是相对轨道角动量、$\boldsymbol{S}^2$是总自旋。对于核物理我们考虑的粒子,单粒子自旋都是$1/2$,因此$\boldsymbol{S}_1^2, \boldsymbol{S}_2^2$略去不写。角动量耦合关系为:
\begin{equation}
\boldsymbol{S}=\boldsymbol{S}_1+\boldsymbol{S}_2,
\end{equation}
\begin{equation}
\boldsymbol{J}=\boldsymbol{L}+\boldsymbol{S}.
\end{equation}

上述力学量的共同本征矢即为所谓的两体分波基矢(LSJ partial-wave basis),求此基矢下的矩阵元的过程就叫做两体相互作用的分波分解。形式上,将分波基矢记为:
\begin{equation}
| p LSJ M_J T T_z \rangle.
\end{equation}

$LSJ$通常称之为一个角动量分波,有一个更加通用的记号${}^{2S+1}\!L_J$,其中$S=0,1$、轨道角动量$L$用对应的字母SPDFGHIJKLMNOX表示,如$^1\!S_0,^3\!P_0$分波等。泡利原理要求额外的$(-1)^{L+S+T}=-1$。

利用角动量耦合,可以将分波基矢进行角动量解耦,得到:
\begin{equation}
\begin{aligned}
| p LSJ M_J T T_z \rangle = &\sum_{M, m_s,t_z} C_{M(M_J-M)M_J}^{LSJ} C^{\frac{1}{2}\frac{1}{2}S}_{m_s(M_J-M-m_s)(M_J-M)} C_{t_z(T_z-t_z)T_z}^{\frac{1}{2}\frac{1}{2}T} \\
& | pLM \rangle \otimes \Big| \frac{1}{2}m_s \Big\rangle \otimes \Big| \frac{1}{2}(M_J-M-m_s) \Big\rangle \otimes \Big| \frac{1}{2}t_z \Big\rangle \otimes \Big| \frac{1}{2}(T_z-t_z) \Big\rangle,
\end{aligned}
\end{equation}
其中由于同位旋部分处理起来非常简单,我们先忽略掉同位旋相关部分,有:
\begin{equation}
\begin{aligned}
| p LSJ M_J \rangle = &\sum_{M, m_s} C_{M(M_J-M)M_J}^{LSJ} C^{\frac{1}{2}\frac{1}{2}S}_{m_s(M_J-M-m_s)(M_J-M)} \\
& | pLM \rangle \otimes \Big| \frac{1}{2}m_s \Big\rangle \otimes \Big| \frac{1}{2}(M_J-M-m_s) \Big\rangle.
\end{aligned}
\end{equation}

借助$| pLM \rangle$在动量空间的波函数:
\begin{equation}
\langle \boldsymbol{p}'|pLM \rangle = \dfrac{\delta(p'-p)}{p'p} Y_{LM}(\hat{\boldsymbol{p}}'),
\end{equation}
可以得到分波表象下相互作用矩阵元的表达式:
\begin{equation}
\begin{aligned}
\langle p' L'S'J' M_{J'} |V_{\mathrm{NN}}| p LSJ M_J \rangle = &\int d\hat{p}' \int d\hat{p} \sum_{M,M',m_s,m_{s'}}\\
& C_{M'(M_{J'}-M')M_{J'}}^{L'S'J'} C_{M(M_J-M)M_J}^{LSJ}\\
& C^{\frac{1}{2}\frac{1}{2}S'}_{m_{s'}(M_{J'}-M'-m_{s'})(M_{J'}-M')} C^{\frac{1}{2}\frac{1}{2}S}_{m_s(M_J-M-m_s)(M_J-M)}\\
& Y_{L'M'}^*(\hat{\boldsymbol{p}}') Y_{LM}(\hat{\boldsymbol{p}})\\
& \langle \boldsymbol{p}' m_{s'} (M_{J'}-M'-m_{s'}) |V_{\mathrm{NN}}| \boldsymbol{p} m_{s} (M_{J}-M-m_{s}) \rangle,
\end{aligned}
\end{equation}
其中最后一部分就是相对坐标系下的平面波矩阵元$\langle \boldsymbol{p}' \sigma_1' \sigma_2' |V_{\mathrm{NN}}| \boldsymbol{p} \sigma_1 \sigma_2 \rangle$,是非常容易计算的。因为两体核力满足$J,M_{J},S$守恒,实际只需要计算:
\begin{equation}
\langle p' L'SJ |V_{\mathrm{NN}}| p LSJ \rangle = \dfrac{1}{2J+1} \sum_{M_J=-J}^{J} \langle p' L'SJ M_{J} |V_{\mathrm{NN}}| p LSJ M_J \rangle,
\end{equation}
由于积分号里面成为了一个标量,我们可以把一个4维积分约化为1维积分,选取:
\begin{equation}
\boldsymbol{p}=(0,0,p),
\end{equation}
\begin{equation}
\boldsymbol{p}'=(p'\sin \theta',0,p'\cos \theta'),
\end{equation}
这样:
\begin{equation}
\int d\hat{p}' \int d\hat{p} = 8\pi^2 \int_{-1}^{1} d(\cos\theta'),
\end{equation}
得到:
\begin{equation}
\begin{aligned}
\langle p' L'S'J' M_{J'} |V_{\mathrm{NN}}| p LSJ M_J \rangle = &8\pi^2 \int_{-1}^{1} d(\cos\theta') \dfrac{1}{2J+1} \sum_{M_J=-J}^{J} \sum_{M'=-L'}^{L'} \sum_{M=-L}^{L} \sum_{m_{s'}=\pm 1/2} \sum_{m_{s}=\pm 1/2}\\
& C_{M'(M_{J'}-M')M_{J'}}^{L'S'J'} C_{M(M_J-M)M_J}^{LSJ}\\
& C^{\frac{1}{2}\frac{1}{2}S'}_{m_{s'}(M_{J'}-M'-m_{s'})(M_{J'}-M')} C^{\frac{1}{2}\frac{1}{2}S}_{m_s(M_J-M-m_s)(M_J-M)}\\
& Y_{L'M'}^*(\theta',0) Y_{LM}(0,0)\\
& \langle \boldsymbol{p}' m_{s'} (M_{J'}-M'-m_{s'}) |V_{\mathrm{NN}}| \boldsymbol{p} m_{s} (M_{J}-M-m_{s}) \rangle.
\end{aligned}
\end{equation}
再借助两体相互作用的算符分解,我们记:
\begin{equation}
h_i(p'pL'LSJ)=\langle p' L'S'J' M_{J'} |w_i| p LSJ M_J \rangle,
\end{equation}
那么最后可以很方便地将任何一个两体相互作用进行分波分解:
\begin{equation}
\langle p' L'S'J' M_{J'} |V_{\mathrm{NN}}| p LSJ M_J \rangle =\sum_i f_i h_i(p'pL'LSJ),
\end{equation}
只需要辨认出前面标量函数$f_i$是什么就足够了。
利用Mathematica,可以很方便地将这样的分波分解过程求出最后解析的表达式,并且转化为代码,极大地提高了计算效率和正确性,这样的技术称之为自动化分波分解[\cite{Golak2010}](automated partial-wave decomposition, aPWD)。我们完整编写了整套自动化分解的代码,并开源在了github上[\cite{code-apwd}]。这种方法相比传统的分波分解具有极大优势,并且可以自然延伸到诸如超核核力、其他更复杂算符、其他自旋粒子等场景中。


\subsection{三体分波表象}
在核物理第一性原理计算中,三体力具有十分特殊且重要的地位。而三体力的分波分解,不仅是少体Faddeev方法、多体计算的必经之路,更是整个\textit{ab initio}方法中最困难、最复杂的技术之一。发展出一个精确、高效、系统的三体力分波分解方法,对于推动整个第一性原理的理论进展具有重要的意义。而上一节的自动化分波投影技术,可以被自然拓展到三体力的情况[\cite{Skibiński2011}]。

三体力的分波矩阵元都是在Jacobi坐标下进行计算并储存的,根据角动量耦合的顺序又分为$LS$-scheme和$JJ$-scheme,简写为$\beta$-scheme和$\alpha$-scheme。我们先在$\beta$-scheme下计算三体力矩阵元,再利用角动量耦合转到$\alpha$-scheme。

在$LS$-scheme中我们定义三体分波基矢为:
\begin{equation}
|\boldsymbol{p} \boldsymbol{q} \beta\rangle \equiv\left|\boldsymbol{p} \boldsymbol{q}(l \lambda) L\left(s \frac{1}{2}\right) S(L S) J M_J\right\rangle
 \otimes
\left|\left(t \frac{1}{2}\right) T m_T\right\rangle,
\end{equation}
其中$\boldsymbol{p}, \boldsymbol{q}$为Jacobi动量;$s,l$是$(23)$核子对的总自旋和相对轨道角动量;$\lambda$是$(23)$核子对与1核子之间的相对轨道角动量;$L$是三体总轨道角动量;$J$是三体总角动量。定义三体态的宇称为:
\begin{equation}
P=(-1)^{l+\lambda}.
\end{equation}

三体力分波矩阵元满足如下守恒条件:
\begin{equation}
\langle \boldsymbol{p}' \boldsymbol{q}' \beta' |V_{\mathrm{3N}}| \boldsymbol{p} \boldsymbol{q} \beta \rangle = \langle \boldsymbol{p}' \boldsymbol{q}' \beta' |V_{\mathrm{3N}}| \boldsymbol{p} \boldsymbol{q} \beta \rangle \delta_{J'J}\delta_{M_{J'}M_{J}}\delta_{P'P}\delta_{T'T}\delta_{M_{T'}M_{T}},
\end{equation}
因此我们后续默认满足这些量子数守恒,并且将矩阵元简记为$G(\beta',\beta)$。

首先单独处理同位旋矩阵元部分,定义:
\begin{equation}
I(t'tT)\equiv \left\langle t' T\right|\hat{I}\left|t T\right\rangle,
\end{equation}
可以解析地计算出矩阵元的表达式,例如:
\begin{equation}
\hat{I}_1=(\boldsymbol{\tau}_2 \cdot \boldsymbol{\tau}_3),
\end{equation}
\begin{equation}
\hat{I}_2=\boldsymbol{\tau}_1\cdot (\boldsymbol{\tau}_2\times \boldsymbol{\tau}_3),
\end{equation}
有:
\begin{equation}
I_1(t'tT)=(2t(t+1)-3)\,\delta_{t,t'},
\end{equation}
\begin{equation}
I_2(t'tT)=2\sqrt{3}i(-1)^{t}\,\delta_{t+t',1}\delta_{T,1/2},
\end{equation}
上述均是由Mathematica符号计算得出,可以保证正确性。

与两体力的分波分解类似,经过角动量耦合的解耦以及分波展开,可以得到:
\begin{equation}
\begin{aligned}
G(\beta',\beta)&=I(t'tT)\cdot\dfrac{1}{2J+1}\sum_{M_J=-J}^{J}\sum_{m_{L'}=-L'}^{L'}\sum_{m_L=-L}^{L}\\
&\cdot C^{L' S' J}_{  m_{L'} (M_J-m_{L'}) M_J} C^{L S J }_{ m_L (M_J-m_L) M_J}\\
&\cdot \int d \hat{p}^{\prime} \int d \hat{q}^{\prime} \int d \hat{p} \int d \hat{q}\;
\mathcal{Y}_{l',\lambda'}^{*L',m_{L'}}(\hat{p}',\hat{q}') \mathcal{Y}_{l,\lambda}^{L,m_{L}}(\hat{p},\hat{q})\\
&\cdot
\left\langle s' S^{\prime} (M_J-m_{L^{\prime}}) \right|V_{\mathrm{3N}}\left(\boldsymbol{p}^{\prime}, \boldsymbol{q}^{\prime}, \boldsymbol{p}, \boldsymbol{q}\right)\left|s S (M_J-m_{L})\right\rangle,
\end{aligned}
\end{equation}
其中耦合球谐函数定义为:
\begin{equation}
\mathcal{Y}_{l,\lambda}^{L,m_{L}}(\hat{p},\hat{q})=\sum_{m_l=-l}^{l}C^{l\lambda L}_{m_l(m_L-m_l)m_L} Y_{l m_l}(\hat{p})Y_{\lambda(m_L-m_l)}(\hat{q}),
\end{equation}
三体自旋耦合态定义为:
\begin{equation}
\begin{aligned}
\left| s S M_S \right\rangle=&
\sum_{m_1=\pm 1/2}\sum_{m_2=\pm 1/2}
C^{\frac{1}{2}sS}_{m_1(M_S-m_1)M_S}
C^{\frac{1}{2}\frac{1}{2}s}_{m_2(M_S-m_1-m_2)(M_S-m_1)}\\
&\left| \frac{1}{2} m_2 \right\rangle \otimes
\left| \frac{1}{2}, M_S-m_1-m_2 \right\rangle \otimes
\left| \frac{1}{2}m_1 \right\rangle.
\end{aligned}
\end{equation}

将积分提出来:
\begin{equation}
G(\beta',\beta)=\int d \hat{p}^{\prime} \int d \hat{q}^{\prime} \int d \hat{p} \int d \hat{q}\;\tilde{G}(\beta',\beta),
\end{equation}
由于积分内是一个标量,可以选取方向:
\begin{equation}
\hat{p}=\hat{z},
\end{equation}
\begin{equation}
\phi_q=0,
\end{equation}
那么:
\begin{equation}
G(\beta',\beta)=8\pi^2 \int \sin \theta_{\hat{q}} \sin \theta_{\hat{p}'} \sin \theta_{\hat{q}'} d\theta_{\hat{q}} d\theta_{\hat{p}'} d\theta_{\hat{q}'} d\phi_{\hat{p}'} d\phi_{\hat{q}'} \;\tilde{G}(\beta',\beta),
\end{equation}
其中$\tilde{G}(\beta',\beta)$可以用Mathematica解析求解出表达式:
\begin{equation}
\begin{aligned}
\tilde{G}(\beta',\beta)&=I(t'tT)\cdot\dfrac{1}{2J+1}\sum_{M_J=-J}^{J}\sum_{m_{L'}=-L'}^{L'}\sum_{m_L=-L}^{L} C^{L' S' J}_{  m_{L'} (M_J-m_{L'}) M_J} C^{L S J }_{ m_L (M_J-m_L) M_J}\\
&\cdot
\mathcal{Y}_{l',\lambda'}^{*L',m_{L'}}(\hat{p}',\hat{q}') \mathcal{Y}_{l,\lambda}^{L,m_{L}}(\hat{p},\hat{q})
\Big\langle s' S^{\prime} (M_J-m_{L^{\prime}}) \Big|V_{\mathrm{3N}}\left(\boldsymbol{p}^{\prime}, \boldsymbol{q}^{\prime}, \boldsymbol{p}, \boldsymbol{q}\right)\Big|s S (M_J-m_{L})\Big\rangle.
\end{aligned}
\end{equation}

在$JJ$-scheme中我们定义三体分波基矢为:
\begin{equation}
|\boldsymbol{p} \boldsymbol{q} \alpha\rangle \equiv\left|\boldsymbol{p} \boldsymbol{q} (ls)j (\lambda\frac{1}{2})j_3 (j j_3)J M_J\right\rangle
 \otimes
\left|\left(t \frac{1}{2}\right) T m_T\right\rangle,
\end{equation}
可以简记为:
\begin{equation}
|\boldsymbol{p} \boldsymbol{q} \alpha\rangle =\left|\boldsymbol{p} \boldsymbol{q} l s j \lambda j_3 t T J \right\rangle,
\end{equation}
它与$LS$-scheme之间通过一个9-j系数联系起来:
\begin{equation}
|\boldsymbol{p} \boldsymbol{q} \alpha\rangle=\sum_{L,S}\sqrt{\hat{L}\hat{S}\hat{j}\hat{j}_3}
\left\{\begin{array}{ccc}
l & s & j \\
\lambda & \frac{1}{2} & j_3 \\
L & S & J
\end{array}\right\}
|\boldsymbol{p} \boldsymbol{q} \beta\rangle,
\end{equation}
因此可以非常方便地求出$LS$-scheme下的三体分波矩阵元$H(\alpha',\alpha)$:
\begin{equation}
H(\alpha',\alpha)=
\sum_{S=1/2}^{3/2}\sum_{L=|J-S|}^{J+S}
\sum_{S'=1/2}^{3/2}\sum_{L'=|J-S'|}^{J+S'}
\sqrt{\hat{L}'\hat{L}\hat{S}'\hat{S}\hat{j}'\hat{j}\hat{j}_3'\hat{j}_3}
\left\{\begin{array}{ccc}
l' & s' & j' \\
\lambda' & \frac{1}{2} & j_3' \\
L' & S' & J
\end{array}\right\}
\left\{\begin{array}{ccc}
l & s & j \\
\lambda & \frac{1}{2} & j_3 \\
L & S & J
\end{array}\right\}
G(\beta',\beta).
\end{equation}

我们同样从头编写了上述三体力分波分解的完整代码,并开源到了github上[\cite{code-apwd3}]。据我们所知,此技术之前一直没有任何开源代码。

值得一提的是,出于中重质量核计算收敛性的要求,通常需要计算大量较高分波的三体力矩阵元,而aPWD方法由于包含了一个5维积分,数值计算上的能力并不是很强。为此,发展新的计算方法是极其有必要的。但目前为止此方面的工作极少,只有K. Hebler通过利用手征三体力的另一个隐藏的对称性将5维积分约化为3维积分,将三体力分波分解的数值计算速度提高了两个数量级,成为目前为止最高效的方法[\cite{PhysRevC.91.044001}]。另外,将程序移植到GPU上进行数值积分也可以极大提高计算效率,是一个值得尝试的方向。

\subsection{两体谐振子表象}

\subsection{三体谐振子表象}

\chapter{手征核力}\label{chap:chiral-force}

\chapter{量子多体方法}\label{chap:many-body}

\chapter{全组态量子蒙特卡洛}\label{chap:fciqmc}
全组态量子蒙特卡洛(Full Configuration-Interaction Quantum Monte Carlo, FCIQMC)是一个组态空间中的量子蒙卡方法,通过在初始态上不断作用虚时投影算符来求解多体系统的基态波函数。

\section{基础算法}
\subsection{虚时投影}
FCIQMC的基本原理就是用虚时演化来间接求解多体薛定谔方程。假设我们从一个与基态非正交的初始波函数$\ket{\Psi(\tau=0)}$出发,将其作用虚时投影算符$\hat{P}(\tau)$:
\begin{equation}
    \ket{\Psi(\tau)}=e^{-\tau\hat{H}}\ket{\Psi(\tau=0)},
\end{equation}
如果我们将初始波函数用$\hat{H}$的本征态进行展开:
\begin{equation}
    \ket{\Psi(\tau=0)}=\sum_{n=0} w_n \ket{\Phi_n},
\end{equation}
那么有:
\begin{equation}
    \ket{\Psi(\tau)}=\sum_{n=0} w_n e^{-\tau E_n}\ket{\Phi_n},
\end{equation}
其中$E_n$是$\ket{\Phi_n}$的本征值。很明显,当虚时间$\tau$逐渐增大时,波函数的各种成分都会呈指数衰减,而基态成分的衰减速度是最慢的。为了防止基态成分衰减为0,我们将虚时投影算符修改为:
\begin{equation}
    \hat{P}(\tau) = e^{-\tau(\hat{H}-E_0)},
\end{equation}
那么除了基态以外,其余激发态成分都会随着$\tau$指数衰减,从而最终得到纯的基态成分:
\begin{equation}
    \ket{\Phi_0}\propto \lim_{\tau\to\infty}\ket{\Psi(\tau)}=e^{-\tau(\hat{H}-E_0)}\ket{\Psi(\tau=0)}.
\end{equation}
将$\hat{P}(\tau)$对虚时间离散化,并作一阶泰勒展开,得到:
\begin{equation}
    \ket{\Psi(\tau+\Delta\tau)}\simeq \big[1-\Delta\tau(\hat{H}-E_0)\big] \ket{\Psi(\tau)},
\end{equation}
其中时间间隔需要满足如下条件来保证泰勒展开的收敛性:
\begin{equation}
    \Delta\tau\leq \dfrac{2}{E_\mathrm{max}-E_\mathrm{min}}.
\end{equation}
其中$E_\mathrm{max}$和$E_\mathrm{min}$分别是系统的最大、最小本征值。注意到,上式对虚时间的离散化虽然引入了一些近似,但是由于近似后的传播子中$\hat{H}$结构的存在,最终投影出来的波函数仍然严格是$\hat{H}$的本征态,也就是在FCIQMC中$\Delta\tau$不会带来任何系统误差。这一特征与其他投影的量子蒙卡方法(如VMC和AFQMC等)具有本质区别。其他量子蒙卡方法对虚时投影算符进行了Suzuki-Trotter分解[\cite{PhysRevC.101.045801}]:
\begin{equation}
    e^{-\Delta\tau(\hat{H}-E_0)}\simeq e^{-\frac{\Delta\tau}{2}\hat{V}} e^{-\Delta\tau \hat{T}} e^{-\frac{\Delta\tau}{2}\hat{V}}+\mathcal{O}(\Delta\tau^3),
\end{equation}
所以需要逐渐减小$\Delta\tau$做多次计算,将结果外推至$\Delta\tau\to 0$来消除这部分系统误差(也称为Trotter误差)。

回到FCIQMC,首先将未知的$E_0$用一个控制变量$S$来代替,然后将波函数在组态空间进行展开:
\begin{equation}
    \ket{\Psi(\tau)}=\sum_i C_i(\tau) \ket{D_i}
\end{equation}
就得到了组态系数的演化方程
\begin{equation}
    C_i(\tau+\Delta\tau) - C_i(\tau) = -\Delta\tau(H_{ii}-S)C_i(\tau) - \Delta\tau\sum_{j\neq i} H_{ij}C_j(\tau)
\end{equation}

现在来定义walker来对组态系数进行蒙卡采样:
\begin{enumerate}
\item 每个walker都位于某个Slater行列式$\ket{D_i}$上,且具有一个自己的方向$\hat{n}$ (由一个单位复向量定义)。
\item 设某个Slater行列式$\ket{D_i}$上有$N_{i}$个walker,若所有walker的方向都同为$\hat{n}_i$,那么定义$\ket{D_i}$上walker的总数目为$\hat{N}_i=N_i \hat{n}_i$,是一个复数。
\item 定义总walker数目为所有组态上walker数目绝对值之和:$N_{\mathrm{w}}=\sum_i |\hat{N}_i|$,是一个实数。
\end{enumerate}

最终FCIQMC的主方程写为:
\begin{equation}
    \hat{N}_i(\tau+\Delta\tau) - \hat{N}_i(\tau) = \textcolor{red}{-\Delta\tau(H_{ii}-S)\hat{N}_i(\tau)} \textcolor{blue}{- \Delta\tau\sum_{j\neq i} H_{ij}\hat{N}_j(\tau)}
\end{equation}

\subsection{演化算法}
FCIQMC的关键之一就是设计一套随机算法,使得walker演化在统计学意义上满足主方程。观察可以发现,主方程的左边就是$\ket{D_i}$上walker总数的变化量,而右边则是对变化量的两种贡献,其中红色和蓝色部分分别是对角项和非对角项的贡献,分别对应下面详细说明的对角死亡步骤和繁殖步骤。

对于每个时间间隔$\Delta\tau$,对所有的walker进行一次演化,算法如下:
\begin{enumerate}
\item \textbf{diagonal step:}遍历所有walker,对于每个walker(设其位于$\ket{D_i}$上,方向为$\hat{n}_i$),计算概率:
    \begin{equation}
        p_\mathrm{death}(i) = \Delta\tau |H_{ii}-S|
    \end{equation}
若$H_{ii}-S>0$,那么这个walker以$p_\mathrm{death}(i)$概率死亡;若$H_{ii}-S<0$,那么这个walker以$p_\mathrm{death}(i)$概率克隆。

\item \textbf{spawning step:}遍历所有walker,对于每个walker(设其位于$\ket{D_i}$上,方向为$\hat{n}_i$),通过激发算法随机选出一个与之连接的组态$\ket{D_f}$,以如下概率在$\ket{D_f}$上产生一个新的walker:
    \begin{equation}
        p_\mathrm{s}(f|i) = \dfrac{\Delta \tau |H_{fi}|}{p_\mathrm{gen}(f|i)}
    \end{equation}
新生成的walker的方向为$-\hat{h}_{fi}\hat{n}_i$,其中$\hat{h}_{fi}=H_{fi}/|H_{fi}|$。

\item \textbf{annihilation step:}收集所有walker,对于位于同一个组态$\ket{D_i}$上的所有walker,将其方向向量求和,得到$\ket{D_i}$的总walker数:$\hat{N}_i=\sum_{\alpha\in\ket{D_i} }\hat{n}_{i,\alpha}=N_i \hat{n}_i$。最后,$\ket{D_i}$上只剩下$N_i$个方向均为$\hat{n}_i$的walker。
\end{enumerate}


\subsection{激发算法}

在FCIQMC中,一个至关重要的算法是spawning step中的激发算法:如何从初态$\ket{D_i}$随机地产生一个末态$\ket{D_s}$。通过对组态相互作用的分析我们知道,对于最高包含三体相互作用的体系,初末态之间只可能相差0,1,2,3个单粒子轨道,而相差0个单粒子轨道就是对角元了,其算法在对角步骤。因此我们仅需考虑三种情况:单激发、双激发算法和三激发。

一个简单的单激发算法:先从$\ket{D_i} = a^\dagger_{\alpha_1} a^\dagger_{\alpha_2} \cdots a^\dagger_{\alpha_n} \ket{0}$中随机选取一个单粒子轨道$\alpha_a$,再将其随机地激发到$\ket{D_i}$中非占据的单粒子轨道$\alpha_b$。但是这个算法有一个问题:$\alpha_b$和$\alpha_a$可能不符合量子数守恒条件,这使得$H_{ji} = \bra{D_j}\hat{H}\ket{D_i} = 0$。虽然不至于造成错误,但是会使得算法空转,严重浪费计算资源。

一个更好的单激发算法是:先从$\ket{D_i}$中随机选取一个单粒子轨道$\alpha_a$,再从非占据轨道中的,满足量子数守恒条件的单粒子轨道中(即所谓单体channel中)随机选取一个组态$\alpha_b$。但是这会产生一个问题:如果当前channel中所有的轨道都是占据的,则没法产生末态。此时的做法是视为正常激发,继续下一个walker的处理。同样,双激发、三激发算法也只在某一个channel中选取激发轨道。下面详细说明激发算法。

当进入激发算法时,随机进入可能的3种激发算法:单激发、双激发、三激发。进入这三种算法的概率满足归一条件:$p_{\mathrm{single}}+p_{\mathrm{double}}+p_{\mathrm{triple}}=1$。

\begin{enumerate}
\item \textbf{single excite:}给定初态$\ket{D_i}$,随机在$\ket{D_i}$的占据态中挑选出一个单粒子轨道$\alpha_a$,可供挑选的单粒子轨道的可能数为$C_A^1$,其中$A$为体系的总核子数;在$\alpha_a$所处的单体channel中进行筛选,找出所有不在$\ket{D_i}$已经占据的轨道中的单粒子态,这些筛选出来的单粒子态共有$N(b|a)$个;在筛选出来的$N(b|a)$个单粒子态中随机选出一个$\alpha_b$,将初态$\ket{D_i}$的$\alpha_a$激发到$\alpha_b$,得到末态$\ket{D_j}$;此时
\begin{equation}
p_{\mathrm{gen}}^{-1}(j|i)=p_{\mathrm{single}}^{-1} C_A^1 N(b|a)
\end{equation}


\item \textbf{double excite:}给定初态$\ket{D_i}$,随机在$\ket{D_i}$的占据态中挑选出两个单粒子轨道$(\alpha_a,\alpha_b)$,可供挑选的两体态的可能数为$C_A^2$;在$(\alpha_a,\alpha_b)$所处的两体channel中进行筛选,找出所有不在$\ket{D_i}$已经占据的轨道中的两体态,这些筛选出来的两体态共有$N(cd|ab)$个;在筛选出来的$N(cd|ab)$个两体态中随机选出一个$(\alpha_c,\alpha_d)$,将初态$\ket{D_i}$的$(\alpha_c,\alpha_d)$激发到$(\alpha_c,\alpha_d)$,得到末态$\ket{D_j}$;此时
\begin{equation}
p_{\mathrm{gen}}^{-1}(j|i)=p_{\mathrm{double}}^{-1} C_A^2 N(cd|ab)
\end{equation}

\item \textbf{triple excite:}给定初态$\ket{D_i}$,随机在$\ket{D_i}$的占据态中挑选出三个单粒子轨道$(\alpha_a,\alpha_b,\alpha_c)$,可供挑选的三体态的可能数为$C_A^3$;在$(\alpha_a,\alpha_b,\alpha_c)$所处的三体channel中进行筛选,找出所有不在$\ket{D_i}$已经占据的轨道中的三体态,这些筛选出来的三体态共有$N(def|abc)$个;在筛选出来的$N(def|abc)$个三体态中随机选出一个$(\alpha_d,\alpha_e,\alpha_f)$,将初态$\ket{D_i}$的$(\alpha_a,\alpha_b,\alpha_c)$激发到$(\alpha_d,\alpha_e,\alpha_f)$,得到末态$\ket{D_j}$;此时
\begin{equation}
p_{\mathrm{gen}}^{-1}(j|i)=p_{\mathrm{triple}}^{-1} C_A^3 N(def|abc)
\end{equation}
\end{enumerate}

可以证明这样的激发算法在统计学意义上可以回到主方程。


\subsection{演化流程}
FCIQMC算法总共分为几个流程:
\begin{enumerate}
\item \textbf{initial condition:}将$N_{\mathrm{ini}}$个walker放置在能量最低的Slater行列式$\ket{D_0}$上,方向向量为$\hat{1}$。
\item \textbf{warm up period:}设定期望的总walker数目为$N_\mathrm{w}$,固定$S=E_0=\bra{D_0}\hat{H}\ket{D_0}$,进行演化算法,walker数目会呈指数增加,直至达到设定值$N_\mathrm{w}$为止。
\item \textbf{projection period:}继续进行演化算法,每$A$步按照如下的方式调整$S$的值,使得总walker数目保持稳定:
\begin{equation}
S(\tau) = S(\tau - \Delta \tau) - \dfrac{\xi}{A\Delta \tau} \ln{\dfrac{N_\mathrm{w}(\tau)}{N_\mathrm{w}(\tau-A\Delta \tau)}} - \dfrac{\zeta}{A\Delta \tau} \ln{\dfrac{N_\mathrm{w}(\tau)}{N_\mathrm{target}}}.
\end{equation}
\item \textbf{statistical period:}当$S(\tau),E(\tau)$大致稳定时,继续进行演化,收集所需要的可观测量的统计数据,最后利用统计学方法求出期望值及统计误差。
\end{enumerate}
在实际计算中,一般选择$N_{\mathrm{ini}}=10,\; A=10,\; \xi=0.1,\; \zeta=\xi^2/4$。

\subsection{可观测量}
在FCIQMC中,直接采样得到的就是波函数的完整组态信息:
\begin{equation}
\ket{\Psi(\tau)}=\sum_i N_i(\tau) \ket{D_i},
\end{equation}
其他所有物理量都可以用波函数来得到。

对于一个可观测量算符$\hat{O}$,若$[\hat{H},\hat{O}]=0$,即$\hat{O}$与$\hat{H}$对易(如哈密顿量算符$\hat{H}$、总角动量算符$\hat{J}^2$),那么其期望值可以通过projected estimator得到:
\begin{equation}
\anglemean{\hat{O}}_{\mathrm{proj}}=\dfrac{\braket{\psi_T}{\hat{O}|\Psi(\tau)}}{\braket{\psi_T}{\Psi(\tau)}}
\end{equation}
只需要保证试探波函数$\ket{\psi_T}$与波函数$\ket{\Psi(\tau)}$有重叠。

对于关联较弱的体系,可以选取能量最低的行列式$\ket{D_0}$作为试探波函数,如:
\begin{equation}
\anglemean{\hat{H}}_{\mathrm{proj}}=\dfrac{1}{N_0(\tau)}\sum_i H_{0i}N_i(\tau).
\end{equation}
可以发现此时波函数中只有与$\ket{D_0}$相连的那些组态才对计算的能量有贡献。并且为了使得计算稳定,需要要求$N_0(\tau)$的值不能太小。

对于组态混合比较明显的体系,选择一个更好的试探波函数能够显著减小统计误差。这样的试探波函数既可以通过其他多体方法计算得到,也可以通过一个$N_\mathrm{w}$较小的FCIQMC预先计算得到。

若$[\hat{H},\hat{O}]\neq 0$,即$\hat{O}$与$\hat{H}$不对易(如半径算符$\hat{r}^2$、密度矩阵),那么其期望值可以通过pure estimator得到:
\begin{equation}
\anglemean{\hat{O}}_{\mathrm{pure}}=\dfrac{\braket{\Psi(\tau)}{\hat{O}|\Psi(\tau)}}{\braket{\Psi(\tau)}{\Psi(\tau)}}
\end{equation}
为了消除pure estimator中的统计误差,需要使用不同随机数种子计算出的两组非关联的波函数来进行计算:
\begin{equation}
\anglemean{\hat{O}}_{\mathrm{pure}}=\dfrac{\braket{\Psi^{(1)}(\tau)}{\hat{O}|\Psi^{(2)}(\tau)}}{\braket{\Psi^{(1)}(\tau)}{\Psi^{(2)}(\tau)}}.
\end{equation}

其他QMC方法会用到mixed estimator (在FCIQMC中我们从来不使用):
\begin{equation}
\anglemean{\hat{O}}_{\mathrm{mix}}=2\anglemean{\hat{O}}_{\mathrm{pure}}-\anglemean{\hat{O}}_{\mathrm{proj}}
\end{equation}

\subsection{起始子近似}
尽管原始版本的FCIQMC算法在小体系中得到了非常成功的应用,但是在大体系的计算中,人们发现需要非常大的walker总数才能够充分压制费米子符号问题,得到收敛的结果。起始子近似(initiator approximation)是将FCIQMC算法应用到较大体系的关键[\cite{10.1063/1.3302277}],它通过对哈密顿量组态矩阵元进行动态截断,很有效地减小了大部分不重要组态对波函数符号产生的噪声。具体的定义如下。

选定一个起始子阈值$n_\alpha$,若某个组态$\ket{D_i}$上的总walker数目满足$|N_i|>n_\alpha$,则这个组态被定义为一个起始子组态。在演化算法的繁殖步骤中中,如果一个非起始子组态尝试向一个未被占据的组态上产生新的walker,这样的尝试被直接拒接。这相当于对哈密顿量采取了如下的截断:
\begin{equation}
\tilde{H}_{ij} =
\begin{cases}
0, & \ket{D_j} \text{ is not initiator and } N_i=0,\\
H_{ij}, & \text{otherwise.}
\end{cases}
\end{equation}
实际计算中我们一般选取$n_\alpha=3$。

i-FCIQMC的主方程实际成为:
\begin{equation}
    \hat{N}_i(\tau+\Delta\tau) - \hat{N}_i(\tau) = -\Delta\tau(H_{ii}-S)\hat{N}_i(\tau) - \Delta\tau\sum_{j\neq i} \tilde{H}_{ij}\hat{N}_j(\tau).
\end{equation}

起始子近似尽管非常有用,但是必然会为计算带来系统误差,需要通过逐渐增加walker总数$N_\mathrm{w}$来系统消除。通过比较不同$n_\alpha$下的计算结果,也可以评估FCIQMC的系统误差。

值得注意的是,上述起始子近似的方案并非唯一。通过借鉴其他方法选取重要组态的方式,可以很好改善起始子近似方案,在保证计算稳定的同时有效减小系统误差。例如,可以利用一个其他的截断的多体方法(如MBPT),获得一个相对可靠的近似波函数,提取出其中占比较大的那些组态,将其设定为起始子,对于其他组态沿用原本近似方案。与之类似,在SHCI以及其他selected CI方法中,有更多更好的组态挑选方案,这对于FCIQMC应该都是有利的[\cite{10.1063/1.5123146}]。


\subsection{自适应偏移}
在大体系的计算中,起始子近似导致的系统误差通常难以通过简单的增加$N_\mathrm{w}$来直接消除。解决这一问题的其中一种方案是自适应偏移算法(adaptive-shift method)[\cite{10.1063/1.5134006,10.1063/5.0032617}]。首先,我们观察到,起始子近似带来系统误差的主要原因在于,非起始子组态的占据情况被系统低估了。假设有一个非起始子组态$\ket{D_i}$,设已占据组态的空间为$\mathcal{A}$,未占据组态的空间为$\mathcal{R}$,那么$\ket{D_i}$向$\mathcal{A}$上繁殖的walkers都被接受了,而向$\mathcal{R}$上繁殖的walkers都被拒绝了。由于$\mathcal{R}$上的占据情况被系统低估,之后$\mathcal{R}$向$\ket{D_i}$的回流(backflow)也会被低估,进一步导致$\ket{D_i}$本身的占据情况被低估。

如果我们已知$\mathcal{R}$中的真实的组态系数,实际上可以将上述缺失的回流贡献补偿到对角项的贡献中:
\begin{equation}
    \hat{N}_i(\tau+\Delta\tau) - \hat{N}_i(\tau) = -\Delta\tau(H_{ii}-S_i^*(\tau))\hat{N}_i(\tau) - \Delta\tau\sum_{j\in \mathcal{A},j\neq i} H_{ij}\hat{N}_j(\tau),
\end{equation}
其中
\begin{equation}
    S_i^*(\tau) = S(\tau)-\sum_{j\in \mathcal{R}} H_{ij}\dfrac{C_j^*}{C_i^*}.
\end{equation}

在实际计算中,我们保持所有起始子组态具有共同的$S(\tau)$,而所有非起始子组态$\ket{D_i}$都具有自己的$S_i(\tau)$:
\begin{equation}
    S_i(\tau) = f_i\times S(\tau),
\end{equation}
其中$f_i$应该反应出这个非起始子组态$\ket{D_i}$被低估的程度。为此,我们对$\ket{D_i}$所发起的繁殖过程进行统计。设某一次向$\ket{D_j}$上尝试了一次繁殖,将这个过程赋予权重:
\begin{equation}
    w_{ij} = \dfrac{|H_{ji}|}{H_{jj}-E_0},
\end{equation}
其中$E_0$为基态能量,可以取为$S$或$E_{\mathrm{proj}}$。最终$f_i$计算为总接受权重的比例:
\begin{equation}
    f_i = \dfrac{\sum_{j\in \text{accepted}} w_{ij}}{\sum_{j\in \text{all}} w_{ij}}.
\end{equation}

为了给自适应偏移方法引入更多自由度,有时也使用如下方式计算非起始子的局域shift:
\begin{equation}
    S_i(\tau) = \Delta + f_i\times[S(\tau)-\Delta],
\end{equation}
其中$\Delta$是一个可调参数。

重要的是,当$N_\mathrm{w}$逐渐增大时,所有组态都会逐渐变为起始子组态,保证了最终仍然会回到最初的主方程。并且通过取不同的$\Delta$,计算出的基态能量随$N_\mathrm{w}$曲线有严格的单调关系,使得真实的基态能量一定会被这些曲线包络起来,因此得到一种估计能量下限的方法,这是其他变分方法无法做到的。

值得注意的是,上述权重的计算方式远非唯一,更加合适的计算方法有可能会进一步加速i-FCIQMC中系统误差的消除。

\subsection{统计分析}
在FCIQMC中,对求出的可观测量进行统计分析时,需要采用分块分析(reblocking analysis)[\cite{10.1063/1.457480}]的方法,而不能简单地直接求标准差,这也是所有MCMC方法的共同要求。下面以基态能量为例,介绍分块分析的方法流程。

\begin{enumerate}
\item 首先,我们在不同时刻求的了$N$个能量的统计值$E_i$。
\item 将$N$个数据均分为$N_L$块,每一个块内包含着$L$个统计值。
\item 对于每一个块$i$,求出这个块内数据的平均值,记为$\bar{E}_i(L)$。
\item 对所有块内平均值$\bar{E}_i(L)$,求出这$N_L$个平均值的平均值和标准差,分别记为$\bar{E}(L)$和$\sigma(L)$。
\item 定义分块后能量的统计误差为$\Delta\bar{E}(L)$,对统计误差的这个估计的误差为$\delta\Delta\bar{E}(L)$:
\begin{equation}
    \Delta\bar{E}(L) = \dfrac{\sigma(L)}{\sqrt{N_L}},
\end{equation}
\begin{equation}
    \delta\Delta\bar{E}(L) = \dfrac{\sigma(L)}{\sqrt{2N_L(N_L-1)}}.
\end{equation}
\item 要求块长度$L$为2的幂次,即$L=2^n$,画出$\Delta\bar{E}(L)$随$L$变化的函数图像,找到$\Delta\bar{E}(L)$的收敛值,作为最终能量标准差的估计值。
\end{enumerate}

分块分析可以内嵌在FCIQMC代码中,使得可以事先指定所需要的统计精度,实现一定的自动化。我们暂时不采取这种方案,实现了一个python版本,公开在[\cite{code-blocking}]。

作为一个例子,我们使用$N_\mathrm{w}=10^4$个walker、$\Delta\mathrm{N^2LO_{GO}}(450)$相互作用,对密度$\rho=0.16$ fm$^{-3}$的纯中子物质进行了计算。对每核子能量的分块分析如图~\ref{fig:analysis}所示。可以看到,对于这个计算,采用$L_0\simeq 10^3$作为块长度来计算标准差,可以得到收敛的估计值。也可以设计一个自动化寻找收敛误差估计的方法,找到$\Delta\bar{E}(L)$随$L$第一次下降发生的$L_0$处作为用于分块分析的最优块长度。

\begin{figure*}[hbtp]
    \centering
    \includegraphics[width=0.6\textwidth]{fig/fig_blocking.pdf}
    \caption{FCIQMC中的分块分析示例计算,其中$\Delta\bar{S}(L)$和$\Delta\bar{E}(L)$分别代表$S$和投影能量的分块分析结果,误差棒代表标准差的误差$\delta\Delta\bar{S}(L)$和$\delta\Delta\bar{E}(L)$。}
    \label{fig:analysis}
\end{figure*}

\subsection{激发态}
为了在FCIQMC中计算激发态,我们需要同时进行一些独立的虚时演化,在每一步演化结束后立刻对波函数进行Gram-Schmidt正交化,来得到正交的波函数[\cite{10.1063/1.4932595,10.1063/1.4986963}]:
\begin{equation}
\ket{\Psi_n} \leftarrow \ket{\Psi_n} -\sum^{n-1}_{m=0}\dfrac{\braket{\Psi^m}{\Psi^n}}{\braket{\Psi^m}{\Psi^m}} \ket{\Psi^m}.
\end{equation}
在FCIQMC中利用这种方式计算激发态时,最好优化初始波函数来加快激发态的收敛速度。


\section{改进算法}
起始子近似和自适应偏移已经成为几乎标准的FCIQMC算法的一部分。在此基础上,还有更多更细化的优化方式,旨在进一步减小统计误差和系统误差。

\subsection{起始子优化}
我们在计算中使用了相干性检验(coherent spawning rule)来减小起始子近似带来的系统误差。具体来说,对于多个非起始子向同一个未占据组态$\ket{D_u}$上的繁殖过程,原本这些过程都是被拒绝的,但是我们计算这些非起始子的总净贡献和绝对贡献之和:
\begin{equation}
W_\mathrm{net} = \sum_{i \text{ from non-initiators}} w_{ui},
\end{equation}
\begin{equation}
W_\mathrm{abs} = \sum_{i \text{ from non-initiators}} |w_{ui}|,
\end{equation}
其中$w_{ui}$是非起始子$\ket{D_i}$尝试向$\ket{D_u}$上产生的walker数目。然后我们计算这些繁殖过程的总符号相干性:
\begin{equation}
\text{Coherence} = W_\mathrm{net}/W_\mathrm{abs},
\end{equation}
若$\text{Coherence}\simeq 0$,表明来自各个非起始子的贡献存在强烈的相消干涉,因此这些贡献都应该被拒绝;若$\text{Coherence}\simeq 1$,表明各个贡献的相位高度一致,这些贡献都应该被接受。具体采用多大的阈值比较合适需要在实际计算中进行测试,我们暂时采用的阈值为0.5。

另外,我们也可以在计算过程中追踪组态与重要组态的连接频率,来改进起始子近似。定义每个组态的年龄为:距离上次连接确定性空间以来所经历的演化步数,即
\begin{equation}
\tau_D = \text{evolution steps since last connect to deterministic space},
\end{equation}
其中确定性空间指的是包含HF组态、HF直接相连组态、起始子组态的集合。

若某个组态的年龄很大,说明其很久没有与重要的组态相连接了,因此调整为非起始子;反之调整为起始子。这种改进的起始子近似可以更加精确地筛选出重要的组态,进一步减小起始子近似带来的系统误差。

\subsection{时间间隔优化}
在QMC计算中,时间间隔$\Delta\tau$不能过大,否则会导致演化过程中大量walker的产生或激发,导致演化不稳定;也不能过小,否则波函数的投影速度比较慢,增加了演化达到平衡所需步长。在量子化学中通常选取的范围为:$10^{-4}$--$10^{-3}$ zs,在核物理计算中通常选取的范围为:$10^{-7}$--$10^{-6}$ zs。这是由于核物理的能标(MeV)比量子化学的能标(keV)高了大约3个数量级,因此时间间隔对应的要小3个数量级。

$\Delta\tau$有严格的上限要求:
\begin{equation}
\Delta\tau < \dfrac{2}{E_{\mathrm{max}}-E_{\mathrm{min}}}.
\end{equation}

虽然实际上无法精确给出$E_{\mathrm{max}}$和$E_{\mathrm{min}}$,但是可以借助Gershgorin圆盘定理或其他矩阵分析方法来很好地给出近似的估计。对于FCIQMC来说,可以简单地采取如下估计:
\begin{equation}
\Delta\tau < \dfrac{2}{E(D_{\mathrm{max}})-E(D_{\mathrm{min}})}
\end{equation}
其中$D_{\mathrm{max}}$和$D_{\mathrm{min}}$分别是单粒子能最高和最低的组态。通常来说,这个估计只能给出粗略的上限,实际计算所需要的$\Delta\tau$一般比这个估计值小一两个数量级。

另外有两个更为细致的要求:
\begin{equation}
\max_{ij} p_{\mathrm{s}}(j|i) < 1
\end{equation}
\begin{equation}
\max_{i} p_{\mathrm{death}}(i) < 1
\end{equation}
第一个要求保证了一个walker不会在其他组态上产生过多walker,导致大量新的组态成为initiator;第二个要求保证了一个walker在对角步骤不会发生符号反转。
在演化过程中动态监控、实时调整时间间隔,可以极大增强FCIQMC演化的稳定性。

上述限制对于绝大多数使用slater行列式的计算来讲都已经足够,但是如果采用对称性限制的其他基矢,需要对上述要求进行修改,用histogram方法保证固定比例(如99.99\%)的spawn事件满足上述条件,详见[\cite{PhysRevB.99.075119}]。我们在程序中都实现了上述算法,并且发现histogram优化的效果是最佳的。


\subsection{激发算法优化}
在激发算法中,发生双激发的可能情况比单激发的多很多,由于它们之间理应地位平等,所以进入双激发的概率也应当比进入单激发的概率高很多,也就是说:
\begin{equation}
p_{\mathrm{single}} \ll p_{\mathrm{double}}
\end{equation}
因此可以选择一个简单的一阶近似:
\begin{equation}
p_{\mathrm{single}} = \dfrac{1}{1+nN},
\end{equation}
\begin{equation}
p_{\mathrm{double}} = 1-p_{\mathrm{single}},
\end{equation}
其中$n$和$N$分别是核子数和单粒子基的数目。

激发算法作为FCIQMC中的最核心算法,是非常灵活的,对于不同物理体系进行合适的调整、优化,也是FCIQMC在不同物理领域获得广泛成功的关键。在基础算法中,在channel中做激发时采取的是均等的概率,这个显然是效率比较低的做法,因为\textit{ab initio}哈密顿量一般是自然稀疏的,对激发过程占主要贡献的激发过程总是比较少的。借助这一先验的知识,我们可以利用哈密顿量矩阵元的数值,来根据重要性挑选激发过程,这就是热浴采样(heat-bath sampling)[\cite{doi:10.1021/acs.jctc.5b01170,10.1063/5.0005754}]。具体来说,在非对角的繁殖步骤中,给定初始组态$\ket{D_i}$,我们令激发算法中挑选出连接组态$\ket{D_j}$的概率正比于组态矩阵元$H_{ji}$:
\begin{equation}
p_{\mathrm{gen}}(f|i) = \dfrac{|H_{fi}|}{\sum_k |H_{ki}|}.
\end{equation}
这在我们的FCIQMC代码中是很容易实现的,因为本来也需要将哈密顿量在m-scheme下的矩阵元分channel存好,所付出的代价只是内存增加了至少一倍。在量子化学中,一般不会显式储存所有的哈密顿量矩阵元,都是现用现算的,因此会进一步对上式采取一些近似。

以双激发过程为例,再详细解释一下热浴采样算法。首先,我们从初始组态$\ket{D_i}$的占据轨道中,以均等概率随机挑选出一对占据轨道$(a,b)$;然后,在$(a,b)$所在两体channel中,以如下概率挑选出一对非占据轨道$(c,d)$:
\begin{equation}
p_{\mathrm{gen}}(cd|ab) = \dfrac{|\braket{cd}{\hat{H}|ab}|}{\sum_{c'd'}|\braket{c'd'}{\hat{H}|ab}|}.
\end{equation}
其中在给定概率分布时进行采样的方法叫做别名采样(alias sampling)[\cite{10.1145/355744.355749}],在计算机、统计学领域有着广泛的应用。我们在代码中自行实现了相应算法,这个算法的效率是$\mathcal{O}(1)$的。

热浴采样算法可以保证贡献最大的那些过程被更加频繁地挑选,使得采样效率得到大幅提高。在实际计算中我们发现,使用热浴采样法进行激发,可以使得计算中的最优$\Delta\tau$增加一到两个数量级,明显缩短了演化达到平衡所需的步数。综合来说,相比于均等概率的激发方式,热浴采样会使得算法的整体效率提升两个数量级以上,因此几乎成为了计算的默认选项。

除此之外,利用对称性来对激发算法进行优化也是一个非常重要的方向。在我们的核物质计算中,通过对哈密顿量矩阵元的分道储存,已经显式考虑了动量守恒和同位旋守恒;在有限核的计算中,则显式考虑了宇称守恒和同位旋守恒。以核物质计算中的双激发过程为例,从一对占据轨道$(a,b)$激发到一对非占据轨道$(c,d)$过程的概率为:
\begin{equation}
p(cd,ab) = p(cd|ab)p(ab),
\end{equation}
其中$p(ab)$是挑选出$(a,b)$的概率。假如我们还想在挑选出$(a,b)$的过程中考虑其他限制,不妨将这些限制条件记为一个集合$\{\mathcal{C}_1,\mathcal{C}_2,\cdots\}$,例如:
\begin{equation}
\mathcal{C}_1 = \text{``nucleon pair have the same spin''},
\end{equation}
\begin{equation}
\mathcal{C}_2 = \text{``nucleon pair have opposite spins''},
\end{equation}
那么有:
\begin{equation}
p(cd,ab) = p(cd,ab|\mathcal{C}_1)p(\mathcal{C}_1) +p(cd,ab|\mathcal{C}_2)p(\mathcal{C}_2),
\end{equation}
\begin{equation}
p(\mathcal{C}_1)+p(\mathcal{C}_2)=1.
\end{equation}
其中:
\begin{equation}
p(cd,ab|\mathcal{C}_1) =p(cd|ab)p(ab|\mathcal{C}_1),
\end{equation}
\begin{equation}
p(cd,ab|\mathcal{C}_2) =p(cd|ab)p(ab|\mathcal{C}_2),
\end{equation}
这使得我们可以在激发过程中按照选定的限制条件进行选择。

\subsection{自旋纯化}
自旋纯化(spin purification)是量子化学中常用的一种提高QMC计算精度的方法,其实现方式有很多,例如可以改造哈密顿量:
\begin{equation}
\hat{H}'=\hat{H}+\lambda \hat{S}^2
\end{equation}
这样就会使得闭壳体系的计算收敛速度变快,并且精度变高,这对于我们的核物质计算应该也是适用的。

对于有限核的计算,我们类似地引入角动量纯化(angular momentum purification)的概念:
\begin{equation}
\hat{H}'=\hat{H}+\lambda \hat{J}^2
\end{equation}
这样对于$J=0$的基态的计算,由于其他角动量的态都被抬高,计算整体的收敛速度也会加快。


\subsection{半随机演化}
对于大多数体系来说,真实的波函数都是由少数占比较大的组态主导的。借助这个先验的知识,可以设计一套半随机演化的算法(semi-stochastic FCIQMC, S-FCIQMC)[\cite{PhysRevLett.109.230201,10.1063/1.4920975}],将全空间分为确定性空间和随机空间,分别记为$\mathcal{D}$和$\mathcal{S}$。在做虚时投影时,在$\mathcal{D}$中用矩阵乘法严格进行虚时投影,而在$\mathcal{S}$中采用通常的FCIQMC算法进行随机的虚时投影:
\begin{equation}
\hat{P} = \hat{P}^{\mathcal{D}} + \hat{P}^{\mathcal{S}},
\end{equation}
其中:
\begin{equation}
\hat{P}^{\mathcal{D}} =\sum_{i\in\mathcal{D},j\in\mathcal{D}} P_{ij}\ket{D_i}\bra{D_j},
\end{equation}

具体的算法流程如下。首先需要确定空间$\mathcal{D}$,它应该由一些重要的组态构成,生成$\mathcal{D}$的方式需要借助一些先验的知识,例如可以简单地规定为HF组态以及与其直接相连的组态,或者在MBPT(2)中有贡献的那些组态。为了使整个FCIQMC方法更加黑箱化,我们先做正常形式的FCIQMC虚时演化,在某一步对所有组态根据占据数的绝对值大小$N_i$进行排序,选择占据最多的那些组态构成$\mathcal{D}$空间,其余组态则都归到$\mathcal{S}$:
\begin{equation}
\mathcal{D} = \max_M \big(\{\ket{D_i}\};|N_i|\big),
\end{equation}
其中$M=|\mathcal{D}|$为$\mathcal{D}$空间的维数。

在虚时演化算法中,对所有占据组态进行遍历时,假设我们遍历到$\ket{D_i}$,分为以下两种情况:
\begin{enumerate}
\item 若$\ket{D_i}\in\mathcal{D}$,则进行正常的随机激发算法。假设某次激发过程中尝试向$\ket{D_j}$繁殖walker,那么这个尝试只有当$\ket{D_j}\in\mathcal{D}$时才能成功,否则这次尝试被拒绝。
\item 若$\ket{D_i}\notin\mathcal{D}$,则进行确定性的激发算法。遍历$\mathcal{D}$中的所有组态$\ket{D_j}\in\mathcal{D}$,直接令$\ket{D_i}$向$\ket{D_j}$上繁殖$-\Delta\tau(H_{ij}-S\delta_{ij})C_j$个walker。
\end{enumerate}
后续对角步骤同样进行,只是对于$\ket{D_i}\notin\mathcal{D}$的情况跳过,因为已经在上面的步骤中做过了。

这样的半随机算法,虽然显著增加了计算量,但是带来的增益是巨大的,既能显著压低随机误差好几个数量级,也有助于减小起始子近似的系统误差、加快波函数的收敛速度。因此,在实际计算中需要尽量使用这个方法。一般来讲,确定空间的维数大约是$|\mathcal{D}|=10^4\sim 10^5$量级。


\subsection{试探波函数}
在FCIQMC中计算投影能量时,一般采用的试探波函数是能量最低的行列式:
\begin{equation}
E_{\mathrm{proj}}=\dfrac{\braket{D_0}{\hat{H}|\Psi(\tau)}}{\braket{D_0}{\Psi(\tau)}}.
\end{equation}
这种评估方式可以通过采用更加精确的试探波函数来改进[\cite{PhysRevLett.109.230201}]。我们首先定义试探空间$\mathcal{T}$,由$N_{\mathcal{T}}$个重要的组态构成。然后将哈密顿量投影到这个子空间,得到投影哈密顿量$H^{\mathcal{T}}$,并且求得子空间内的本征态和本征值:
\begin{equation}
\ket{\Psi^{\mathcal{T}}} = \sum_{i\in \mathcal{T}} C_{i}^{\mathcal{T}} \ket{D_i},
\end{equation}
\begin{equation}
H^{\mathcal{T}} \ket{\Psi^{\mathcal{T}}} = E^{\mathcal{T}}\ket{\Psi^{\mathcal{T}}}.
\end{equation}
最后,在计算投影能量时,有:
\begin{equation}
E_{\mathrm{trial}} =\dfrac{\braket{\Psi^{\mathcal{T}}}{\hat{H}|\Psi(\tau)}}{\braket{\Psi^{\mathcal{T}}}{\Psi(\tau)}} =\dfrac{\sum_{i\in\mathcal{C}}C_iV_i}{\sum_{i\in\mathcal{T}}C_iC_i^{\mathcal{T}*}},
\end{equation}
其中$\mathcal{C}$定义为与$\mathcal{T}$通过一个哈密顿量算符$\hat{H}$相连的所有组态构成的空间,且
\begin{equation}
V_i= \sum_{j\in\mathcal{T}} \braket{D_j}{\hat{H}|D_i} C_j^{\mathcal{T}*}.
\end{equation}
因为$\mathcal{C}$的维数比$\mathcal{T}$的要大很多,因此这个方法中增加的计算量主要来源于向量$V_i$的相关计算。我们推荐在实际计算中采用的试探空间的维数大约为$|\mathcal{T}|\sim10^2$。对于激发态或组态混合比较强的基态,利用试探波函数计算投影能量可以显著减小统计误差和系统误差。

最后我们用一个实际的例子来展示S-FCIQMC和试探波函数的效果。使用了相互作用$\Delta\mathrm{N^2LO_{GO}} (450)$计算38个核子的纯中子物质,使用$N_\mathrm{w}=10^5$个walker,前面先做正常的全随机的FCIQMC虚时演化,然后在10000步开始做半随机的S-FCIQMC演化,并且使用试探波函数来计算投影能量,其中$|\mathcal{D}|=5000$,$|\mathcal{T}|=100$,投影能量的统计情况如图[\ref{fig:semistochastic}]所示。很明显,改进算法可以显著增强计算的稳定性。

\begin{figure*}[hbtp]
    \centering
    \includegraphics[width=0.6\textwidth]{fig/fig_semistochastic.pdf}
    \caption{S-FCIQMC中的示例计算,其中前半部分使用了全随机的FCIQMC演化算法,后半部分使用了半随机的S-FCIQMC演化算法,并且使用了100维的试探波函数。}
    \label{fig:semistochastic}
\end{figure*}

% \chapter{核物质的精确计算}\label{chap:matter}

% \chapter{有限核的精确计算}\label{chap:nuclei}

% \chapter{有限温方法}\label{chap:ft}

% \chapter{混合多体方法}\label{chap:hybrid}

% \chapter{本征矢延拓}\label{chap:EC}

\specialchap{总结与展望}


% Copyright (c) 2014,2016 Casper Ti. Vector
% Public domain.

\specialchap{攻读博士学位期间发表的论文}


\begin{enumerate}

\item Z. C. Xu, \textbf{\underline{R. Z. Hu}}, S. L. Jin, J. H. Hou, S. Zhang, and F. R. Xu$^*$, Collectivity of nuclei near the exotic doubly magic Ni 78 by ab initio calculations, Phys. Rev. C 110, 024308 (2024).

\end{enumerate}
\appendix
\chapter{物理常数}

\chapter{Richardson模型}

Richardson模型的哈密顿量 [\cite{RevModPhys.76.643}]:
\begin{equation}
\hat{H}=\delta \sum_{p=1}^{p_\mathrm{max}} \sum_{\sigma=\uparrow,\downarrow} (p-1)a_{p\sigma}^\dagger a_{p\sigma} -\dfrac{g}{2} \sum_{p,q=1}^{p_\mathrm{max}} a_{p\uparrow}^\dagger a_{p\downarrow}^\dagger a_{q\uparrow} a_{q\downarrow}
\end{equation}
不失一般性,令$\delta=1.0$,并选取$p_\mathrm{max}=4$,$A=4$
的半满情况进行benchmark。

MBPT的各阶贡献有解析表达式:
\begin{equation}
E_{\mathrm{MBPT}}^{(2)} = -\frac{g^2 \left(g^2+8 g+14\right)}{g^3+12 g^2+44 g+48}
\end{equation}
\begin{equation}
E_{\mathrm{MBPT}}^{(3)} = -\frac{2 g^3 \left(g^4+16 g^3+93 g^2+232 g+212\right)}{\left(g^3+12 g^2+44
   g+48\right)^2}
\end{equation}
\begin{equation}
E_{\mathrm{MBPT}}^{(4)} = -\frac{g^4 \left(7 g^6+168 g^5+1652 g^4+8512 g^3+24252 g^2+36320
   g+22416\right)}{2 \left(g^3+12 g^2+44 g+48\right)^3}
\end{equation}
MBPT的各阶修正在$g=-6,-4,-2$处是发散的。
\printbibliography[heading=bibintoc]
\backmatter
\chapter{致谢}

感谢。

\vspace{2cm}
\hfill{2026年07月23日于加速器楼}

\input{src-app/originauth}


\end{document}
