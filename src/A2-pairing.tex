\chapter{Richardson模型}

Richardson模型的哈密顿量 [\cite{RevModPhys.76.643}]:
\begin{equation}
\hat{H}=\delta \sum_{p=1}^{p_\mathrm{max}} \sum_{\sigma=\uparrow,\downarrow} (p-1)a_{p\sigma}^\dagger a_{p\sigma} -\dfrac{g}{2} \sum_{p,q=1}^{p_\mathrm{max}} a_{p\uparrow}^\dagger a_{p\downarrow}^\dagger a_{q\uparrow} a_{q\downarrow}
\end{equation}
不失一般性,令$\delta=1.0$,并选取$p_\mathrm{max}=4$,$A=4$
的半满情况进行benchmark。

MBPT的各阶贡献有解析表达式:
\begin{equation}
E_{\mathrm{MBPT}}^{(2)} = -\frac{g^2 \left(g^2+8 g+14\right)}{g^3+12 g^2+44 g+48}
\end{equation}
\begin{equation}
E_{\mathrm{MBPT}}^{(3)} = -\frac{2 g^3 \left(g^4+16 g^3+93 g^2+232 g+212\right)}{\left(g^3+12 g^2+44
   g+48\right)^2}
\end{equation}
\begin{equation}
E_{\mathrm{MBPT}}^{(4)} = -\frac{g^4 \left(7 g^6+168 g^5+1652 g^4+8512 g^3+24252 g^2+36320
   g+22416\right)}{2 \left(g^3+12 g^2+44 g+48\right)^3}
\end{equation}
MBPT的各阶修正在$g=-6,-4,-2$处是发散的。