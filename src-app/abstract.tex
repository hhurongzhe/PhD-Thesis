\begin{cabstract}
    原子核结构的第一性原理计算({\it ab initio})是指从现实核力出发,运用严格的量子多体方法求解原子核结构问题。本论文对多体微扰理论及壳模型计算进行了如下发展:
\begin{enumerate}
    \item \textbf{全组态相互作用量子蒙特卡洛}\; 壳模型计算是在一个较大的组态空间对哈密顿量直接对角化,由于组态空间维度随粒子增加而迅速增大,其难以计算较多核子数的体系,比如无核芯壳模型仅适用于轻核的计算。全组态相互作用量子蒙特卡洛(full-configuration interaction Quantum Monte Carlo, FCIQMC)方法是在电子结构计算中获得成功的一种组态空间计算方法,其通过对组态空间波函数进行蒙特卡洛采样,能够一定程度上解决组态空间维度过大的问题。我们成功将这一方法应用到原子核结构计算中,并通过$^{56}\mathrm{Ni}$的计算,表明这一方法能够比耦合团簇方法考虑更多的多体关联。
\end{enumerate}
\end{cabstract}

\begin{eabstract}
First-principles ({\it ab initio}) nuclear structure calculations refer to solving nuclear structure problems using rigorous quantum many-body methods starting from realistic nuclear forces. This thesis presents the following advancements in MBPT and shell model calculations:

\begin{enumerate}
    \item \textbf{Full-Configuration Interaction Quantum Monte Carlo}\; The shell model involves direct diagonalization of Hamiltonians in large configuration spaces. Due to rapid dimension growth with increasing nucleon numbers, it becomes challenging for systems with more nucleons (e.g., no-core shell models are limited to light nuclei). The full-configuration interaction Quantum Monte Carlo (FCIQMC) method, successful in electronic structure calculations, addresses the issue of configuration space dimensionality through Monte Carlo sampling of the configuration-space wavefunction. We successfully implemented this method in nuclear structure calculations. Through $^{56}\mathrm{Ni}$ calculations, we demonstrated that FCIQMC can account for more extensive many-body correlations compared to coupled-cluster methods.

\end{enumerate}
\end{eabstract}